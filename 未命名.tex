% Options for packages loaded elsewhere
\PassOptionsToPackage{unicode}{hyperref}
\PassOptionsToPackage{hyphens}{url}
%
\documentclass[
]{article}
\usepackage{amsmath,amssymb}
\usepackage{iftex}
\ifPDFTeX
  \usepackage[T1]{fontenc}
  \usepackage[utf8]{inputenc}
  \usepackage{textcomp} % provide euro and other symbols
\else % if luatex or xetex
  \usepackage{unicode-math} % this also loads fontspec
  \defaultfontfeatures{Scale=MatchLowercase}
  \defaultfontfeatures[\rmfamily]{Ligatures=TeX,Scale=1}
\fi
\usepackage{lmodern}
\ifPDFTeX\else
  % xetex/luatex font selection
\fi
% Use upquote if available, for straight quotes in verbatim environments
\IfFileExists{upquote.sty}{\usepackage{upquote}}{}
\IfFileExists{microtype.sty}{% use microtype if available
  \usepackage[]{microtype}
  \UseMicrotypeSet[protrusion]{basicmath} % disable protrusion for tt fonts
}{}
\makeatletter
\@ifundefined{KOMAClassName}{% if non-KOMA class
  \IfFileExists{parskip.sty}{%
    \usepackage{parskip}
  }{% else
    \setlength{\parindent}{0pt}
    \setlength{\parskip}{6pt plus 2pt minus 1pt}}
}{% if KOMA class
  \KOMAoptions{parskip=half}}
\makeatother
\usepackage{xcolor}
\usepackage[margin=1in]{geometry}
\usepackage{color}
\usepackage{fancyvrb}
\newcommand{\VerbBar}{|}
\newcommand{\VERB}{\Verb[commandchars=\\\{\}]}
\DefineVerbatimEnvironment{Highlighting}{Verbatim}{commandchars=\\\{\}}
% Add ',fontsize=\small' for more characters per line
\usepackage{framed}
\definecolor{shadecolor}{RGB}{248,248,248}
\newenvironment{Shaded}{\begin{snugshade}}{\end{snugshade}}
\newcommand{\AlertTok}[1]{\textcolor[rgb]{0.94,0.16,0.16}{#1}}
\newcommand{\AnnotationTok}[1]{\textcolor[rgb]{0.56,0.35,0.01}{\textbf{\textit{#1}}}}
\newcommand{\AttributeTok}[1]{\textcolor[rgb]{0.13,0.29,0.53}{#1}}
\newcommand{\BaseNTok}[1]{\textcolor[rgb]{0.00,0.00,0.81}{#1}}
\newcommand{\BuiltInTok}[1]{#1}
\newcommand{\CharTok}[1]{\textcolor[rgb]{0.31,0.60,0.02}{#1}}
\newcommand{\CommentTok}[1]{\textcolor[rgb]{0.56,0.35,0.01}{\textit{#1}}}
\newcommand{\CommentVarTok}[1]{\textcolor[rgb]{0.56,0.35,0.01}{\textbf{\textit{#1}}}}
\newcommand{\ConstantTok}[1]{\textcolor[rgb]{0.56,0.35,0.01}{#1}}
\newcommand{\ControlFlowTok}[1]{\textcolor[rgb]{0.13,0.29,0.53}{\textbf{#1}}}
\newcommand{\DataTypeTok}[1]{\textcolor[rgb]{0.13,0.29,0.53}{#1}}
\newcommand{\DecValTok}[1]{\textcolor[rgb]{0.00,0.00,0.81}{#1}}
\newcommand{\DocumentationTok}[1]{\textcolor[rgb]{0.56,0.35,0.01}{\textbf{\textit{#1}}}}
\newcommand{\ErrorTok}[1]{\textcolor[rgb]{0.64,0.00,0.00}{\textbf{#1}}}
\newcommand{\ExtensionTok}[1]{#1}
\newcommand{\FloatTok}[1]{\textcolor[rgb]{0.00,0.00,0.81}{#1}}
\newcommand{\FunctionTok}[1]{\textcolor[rgb]{0.13,0.29,0.53}{\textbf{#1}}}
\newcommand{\ImportTok}[1]{#1}
\newcommand{\InformationTok}[1]{\textcolor[rgb]{0.56,0.35,0.01}{\textbf{\textit{#1}}}}
\newcommand{\KeywordTok}[1]{\textcolor[rgb]{0.13,0.29,0.53}{\textbf{#1}}}
\newcommand{\NormalTok}[1]{#1}
\newcommand{\OperatorTok}[1]{\textcolor[rgb]{0.81,0.36,0.00}{\textbf{#1}}}
\newcommand{\OtherTok}[1]{\textcolor[rgb]{0.56,0.35,0.01}{#1}}
\newcommand{\PreprocessorTok}[1]{\textcolor[rgb]{0.56,0.35,0.01}{\textit{#1}}}
\newcommand{\RegionMarkerTok}[1]{#1}
\newcommand{\SpecialCharTok}[1]{\textcolor[rgb]{0.81,0.36,0.00}{\textbf{#1}}}
\newcommand{\SpecialStringTok}[1]{\textcolor[rgb]{0.31,0.60,0.02}{#1}}
\newcommand{\StringTok}[1]{\textcolor[rgb]{0.31,0.60,0.02}{#1}}
\newcommand{\VariableTok}[1]{\textcolor[rgb]{0.00,0.00,0.00}{#1}}
\newcommand{\VerbatimStringTok}[1]{\textcolor[rgb]{0.31,0.60,0.02}{#1}}
\newcommand{\WarningTok}[1]{\textcolor[rgb]{0.56,0.35,0.01}{\textbf{\textit{#1}}}}
\usepackage{graphicx}
\makeatletter
\def\maxwidth{\ifdim\Gin@nat@width>\linewidth\linewidth\else\Gin@nat@width\fi}
\def\maxheight{\ifdim\Gin@nat@height>\textheight\textheight\else\Gin@nat@height\fi}
\makeatother
% Scale images if necessary, so that they will not overflow the page
% margins by default, and it is still possible to overwrite the defaults
% using explicit options in \includegraphics[width, height, ...]{}
\setkeys{Gin}{width=\maxwidth,height=\maxheight,keepaspectratio}
% Set default figure placement to htbp
\makeatletter
\def\fps@figure{htbp}
\makeatother
\setlength{\emergencystretch}{3em} % prevent overfull lines
\providecommand{\tightlist}{%
  \setlength{\itemsep}{0pt}\setlength{\parskip}{0pt}}
\setcounter{secnumdepth}{-\maxdimen} % remove section numbering
\ifLuaTeX
  \usepackage{selnolig}  % disable illegal ligatures
\fi
\usepackage{bookmark}
\IfFileExists{xurl.sty}{\usepackage{xurl}}{} % add URL line breaks if available
\urlstyle{same}
\hypersetup{
  pdftitle={Untitled},
  pdfauthor={Tai Yue},
  hidelinks,
  pdfcreator={LaTeX via pandoc}}

\title{Untitled}
\author{Tai Yue}
\date{2024-11-11}

\begin{document}
\maketitle

problem1

\begin{Shaded}
\begin{Highlighting}[]
\NormalTok{has\_shared\_birthday }\OtherTok{\textless{}{-}} \ControlFlowTok{function}\NormalTok{(n) \{}
 
\NormalTok{  birthdays }\OtherTok{\textless{}{-}} \FunctionTok{sample}\NormalTok{(}\DecValTok{1}\SpecialCharTok{:}\DecValTok{365}\NormalTok{, n, }\AttributeTok{replace =} \ConstantTok{TRUE}\NormalTok{)}
  
  
  \FunctionTok{return}\NormalTok{(}\FunctionTok{length}\NormalTok{(birthdays) }\SpecialCharTok{!=} \FunctionTok{length}\NormalTok{(}\FunctionTok{unique}\NormalTok{(birthdays)))}
\NormalTok{\}}
\end{Highlighting}
\end{Shaded}

\begin{Shaded}
\begin{Highlighting}[]
\NormalTok{has\_shared\_birthday }\OtherTok{\textless{}{-}} \ControlFlowTok{function}\NormalTok{(n) \{}
\NormalTok{  birthdays }\OtherTok{\textless{}{-}} \FunctionTok{sample}\NormalTok{(}\DecValTok{1}\SpecialCharTok{:}\DecValTok{365}\NormalTok{, n, }\AttributeTok{replace =} \ConstantTok{TRUE}\NormalTok{)}
  \FunctionTok{return}\NormalTok{(}\FunctionTok{length}\NormalTok{(birthdays) }\SpecialCharTok{!=} \FunctionTok{length}\NormalTok{(}\FunctionTok{unique}\NormalTok{(birthdays)))}
\NormalTok{\}}


\NormalTok{group\_sizes }\OtherTok{\textless{}{-}} \DecValTok{2}\SpecialCharTok{:}\DecValTok{50}
\NormalTok{n\_simulations }\OtherTok{\textless{}{-}} \DecValTok{10000}
\NormalTok{probabilities }\OtherTok{\textless{}{-}} \FunctionTok{numeric}\NormalTok{(}\FunctionTok{length}\NormalTok{(group\_sizes))}


\ControlFlowTok{for}\NormalTok{ (i }\ControlFlowTok{in} \FunctionTok{seq\_along}\NormalTok{(group\_sizes)) \{}
\NormalTok{  n }\OtherTok{\textless{}{-}}\NormalTok{ group\_sizes[i]}
\NormalTok{  duplicates }\OtherTok{\textless{}{-}} \FunctionTok{sum}\NormalTok{(}\FunctionTok{replicate}\NormalTok{(n\_simulations, }\FunctionTok{has\_shared\_birthday}\NormalTok{(n)))}
\NormalTok{  probabilities[i] }\OtherTok{\textless{}{-}}\NormalTok{ duplicates }\SpecialCharTok{/}\NormalTok{ n\_simulations}
\NormalTok{\}}


\FunctionTok{plot}\NormalTok{(group\_sizes, probabilities, }\AttributeTok{type =} \StringTok{"o"}\NormalTok{, }\AttributeTok{pch =} \DecValTok{19}\NormalTok{,}
     \AttributeTok{xlab =} \StringTok{"Group Size"}\NormalTok{, }\AttributeTok{ylab =} \StringTok{"Probability of Shared Birthday"}\NormalTok{,}
     \AttributeTok{main =} \StringTok{"Probability of Shared Birthday as a Function of Group Size"}\NormalTok{)}
\FunctionTok{grid}\NormalTok{()}
\end{Highlighting}
\end{Shaded}

\includegraphics{未命名_files/figure-latex/unnamed-chunk-2-1.pdf}

problem2:

\begin{Shaded}
\begin{Highlighting}[]
\FunctionTok{library}\NormalTok{(broom)}
\FunctionTok{library}\NormalTok{(ggplot2)}


\NormalTok{n }\OtherTok{\textless{}{-}} \DecValTok{30}
\NormalTok{sigma }\OtherTok{\textless{}{-}} \DecValTok{5}
\NormalTok{alpha }\OtherTok{\textless{}{-}} \FloatTok{0.05}
\NormalTok{mu\_values }\OtherTok{\textless{}{-}} \FunctionTok{c}\NormalTok{(}\DecValTok{0}\NormalTok{, }\DecValTok{1}\NormalTok{, }\DecValTok{2}\NormalTok{, }\DecValTok{3}\NormalTok{, }\DecValTok{4}\NormalTok{, }\DecValTok{5}\NormalTok{, }\DecValTok{6}\NormalTok{)}
\NormalTok{num\_simulations }\OtherTok{\textless{}{-}} \DecValTok{5000}


\NormalTok{results }\OtherTok{\textless{}{-}} \FunctionTok{data.frame}\NormalTok{()}

\ControlFlowTok{for}\NormalTok{ (mu }\ControlFlowTok{in}\NormalTok{ mu\_values) \{}
  
\NormalTok{  rejections }\OtherTok{\textless{}{-}} \DecValTok{0}
  
  \ControlFlowTok{for}\NormalTok{ (i }\ControlFlowTok{in} \DecValTok{1}\SpecialCharTok{:}\NormalTok{num\_simulations) \{}
   
\NormalTok{    sample }\OtherTok{\textless{}{-}} \FunctionTok{rnorm}\NormalTok{(n, }\AttributeTok{mean =}\NormalTok{ mu, }\AttributeTok{sd =}\NormalTok{ sigma)}
    
    
\NormalTok{    t\_test\_result }\OtherTok{\textless{}{-}} \FunctionTok{t.test}\NormalTok{(sample, }\AttributeTok{mu =} \DecValTok{0}\NormalTok{)}
\NormalTok{    tidy\_result }\OtherTok{\textless{}{-}} \FunctionTok{tidy}\NormalTok{(t\_test\_result)}
    
    
    \ControlFlowTok{if}\NormalTok{ (tidy\_result}\SpecialCharTok{$}\NormalTok{p.value }\SpecialCharTok{\textless{}}\NormalTok{ alpha) \{}
\NormalTok{      rejections }\OtherTok{\textless{}{-}}\NormalTok{ rejections }\SpecialCharTok{+} \DecValTok{1}
\NormalTok{    \}}
\NormalTok{  \}}
  
  
\NormalTok{  power }\OtherTok{\textless{}{-}}\NormalTok{ rejections }\SpecialCharTok{/}\NormalTok{ num\_simulations}
  
  
\NormalTok{  results }\OtherTok{\textless{}{-}} \FunctionTok{rbind}\NormalTok{(results, }\FunctionTok{data.frame}\NormalTok{(}\AttributeTok{mu =}\NormalTok{ mu, }\AttributeTok{power =}\NormalTok{ power))}
\NormalTok{\}}


\FunctionTok{ggplot}\NormalTok{(results, }\FunctionTok{aes}\NormalTok{(}\AttributeTok{x =}\NormalTok{ mu, }\AttributeTok{y =}\NormalTok{ power)) }\SpecialCharTok{+}
  \FunctionTok{geom\_line}\NormalTok{() }\SpecialCharTok{+}
  \FunctionTok{geom\_point}\NormalTok{() }\SpecialCharTok{+}
  \FunctionTok{labs}\NormalTok{(}
    \AttributeTok{x =} \StringTok{"True Value of μ"}\NormalTok{,}
    \AttributeTok{y =} \StringTok{"Power (Proportion of Null Rejections)"}\NormalTok{,}
    \AttributeTok{title =} \StringTok{"Power of One{-}Sample t{-}Test vs. Effect Size (μ)"}
\NormalTok{  ) }\SpecialCharTok{+}
  \FunctionTok{theme\_minimal}\NormalTok{()}
\end{Highlighting}
\end{Shaded}

\includegraphics{未命名_files/figure-latex/unnamed-chunk-3-1.pdf}

For small values of μ, the power is low, meaning the test often fails to
reject the null hypothesis. As μ increases, the power rises
significantly, reaching close to 1 for μ values around 4 and higher.
This means that, with larger effect sizes, the test almost always
correctly rejects the null hypothesis.The trend suggests a positive
relationship between effect size and power: larger effect sizes make it
easier to detect an effect, thus increasing the test's power.

\end{document}
